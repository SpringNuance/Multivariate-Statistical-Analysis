% Options for packages loaded elsewhere
\PassOptionsToPackage{unicode}{hyperref}
\PassOptionsToPackage{hyphens}{url}
\PassOptionsToPackage{dvipsnames,svgnames,x11names}{xcolor}
%
\documentclass[
]{article}
\title{Assignment 1}
\author{Nguyen Xuan Binh}
\date{}

\usepackage{amsmath,amssymb}
\usepackage{lmodern}
\usepackage{iftex}
\ifPDFTeX
  \usepackage[T1]{fontenc}
  \usepackage[utf8]{inputenc}
  \usepackage{textcomp} % provide euro and other symbols
\else % if luatex or xetex
  \usepackage{unicode-math}
  \defaultfontfeatures{Scale=MatchLowercase}
  \defaultfontfeatures[\rmfamily]{Ligatures=TeX,Scale=1}
\fi
% Use upquote if available, for straight quotes in verbatim environments
\IfFileExists{upquote.sty}{\usepackage{upquote}}{}
\IfFileExists{microtype.sty}{% use microtype if available
  \usepackage[]{microtype}
  \UseMicrotypeSet[protrusion]{basicmath} % disable protrusion for tt fonts
}{}
\makeatletter
\@ifundefined{KOMAClassName}{% if non-KOMA class
  \IfFileExists{parskip.sty}{%
    \usepackage{parskip}
  }{% else
    \setlength{\parindent}{0pt}
    \setlength{\parskip}{6pt plus 2pt minus 1pt}}
}{% if KOMA class
  \KOMAoptions{parskip=half}}
\makeatother
\usepackage{xcolor}
\IfFileExists{xurl.sty}{\usepackage{xurl}}{} % add URL line breaks if available
\IfFileExists{bookmark.sty}{\usepackage{bookmark}}{\usepackage{hyperref}}
\hypersetup{
  pdftitle={Assignment 1},
  pdfauthor={Nguyen Xuan Binh},
  colorlinks=true,
  linkcolor={Maroon},
  filecolor={Maroon},
  citecolor={Blue},
  urlcolor={blue},
  pdfcreator={LaTeX via pandoc}}
\urlstyle{same} % disable monospaced font for URLs
\usepackage[margin=1in]{geometry}
\usepackage{color}
\usepackage{fancyvrb}
\newcommand{\VerbBar}{|}
\newcommand{\VERB}{\Verb[commandchars=\\\{\}]}
\DefineVerbatimEnvironment{Highlighting}{Verbatim}{commandchars=\\\{\}}
% Add ',fontsize=\small' for more characters per line
\usepackage{framed}
\definecolor{shadecolor}{RGB}{248,248,248}
\newenvironment{Shaded}{\begin{snugshade}}{\end{snugshade}}
\newcommand{\AlertTok}[1]{\textcolor[rgb]{0.94,0.16,0.16}{#1}}
\newcommand{\AnnotationTok}[1]{\textcolor[rgb]{0.56,0.35,0.01}{\textbf{\textit{#1}}}}
\newcommand{\AttributeTok}[1]{\textcolor[rgb]{0.77,0.63,0.00}{#1}}
\newcommand{\BaseNTok}[1]{\textcolor[rgb]{0.00,0.00,0.81}{#1}}
\newcommand{\BuiltInTok}[1]{#1}
\newcommand{\CharTok}[1]{\textcolor[rgb]{0.31,0.60,0.02}{#1}}
\newcommand{\CommentTok}[1]{\textcolor[rgb]{0.56,0.35,0.01}{\textit{#1}}}
\newcommand{\CommentVarTok}[1]{\textcolor[rgb]{0.56,0.35,0.01}{\textbf{\textit{#1}}}}
\newcommand{\ConstantTok}[1]{\textcolor[rgb]{0.00,0.00,0.00}{#1}}
\newcommand{\ControlFlowTok}[1]{\textcolor[rgb]{0.13,0.29,0.53}{\textbf{#1}}}
\newcommand{\DataTypeTok}[1]{\textcolor[rgb]{0.13,0.29,0.53}{#1}}
\newcommand{\DecValTok}[1]{\textcolor[rgb]{0.00,0.00,0.81}{#1}}
\newcommand{\DocumentationTok}[1]{\textcolor[rgb]{0.56,0.35,0.01}{\textbf{\textit{#1}}}}
\newcommand{\ErrorTok}[1]{\textcolor[rgb]{0.64,0.00,0.00}{\textbf{#1}}}
\newcommand{\ExtensionTok}[1]{#1}
\newcommand{\FloatTok}[1]{\textcolor[rgb]{0.00,0.00,0.81}{#1}}
\newcommand{\FunctionTok}[1]{\textcolor[rgb]{0.00,0.00,0.00}{#1}}
\newcommand{\ImportTok}[1]{#1}
\newcommand{\InformationTok}[1]{\textcolor[rgb]{0.56,0.35,0.01}{\textbf{\textit{#1}}}}
\newcommand{\KeywordTok}[1]{\textcolor[rgb]{0.13,0.29,0.53}{\textbf{#1}}}
\newcommand{\NormalTok}[1]{#1}
\newcommand{\OperatorTok}[1]{\textcolor[rgb]{0.81,0.36,0.00}{\textbf{#1}}}
\newcommand{\OtherTok}[1]{\textcolor[rgb]{0.56,0.35,0.01}{#1}}
\newcommand{\PreprocessorTok}[1]{\textcolor[rgb]{0.56,0.35,0.01}{\textit{#1}}}
\newcommand{\RegionMarkerTok}[1]{#1}
\newcommand{\SpecialCharTok}[1]{\textcolor[rgb]{0.00,0.00,0.00}{#1}}
\newcommand{\SpecialStringTok}[1]{\textcolor[rgb]{0.31,0.60,0.02}{#1}}
\newcommand{\StringTok}[1]{\textcolor[rgb]{0.31,0.60,0.02}{#1}}
\newcommand{\VariableTok}[1]{\textcolor[rgb]{0.00,0.00,0.00}{#1}}
\newcommand{\VerbatimStringTok}[1]{\textcolor[rgb]{0.31,0.60,0.02}{#1}}
\newcommand{\WarningTok}[1]{\textcolor[rgb]{0.56,0.35,0.01}{\textbf{\textit{#1}}}}
\usepackage{graphicx}
\makeatletter
\def\maxwidth{\ifdim\Gin@nat@width>\linewidth\linewidth\else\Gin@nat@width\fi}
\def\maxheight{\ifdim\Gin@nat@height>\textheight\textheight\else\Gin@nat@height\fi}
\makeatother
% Scale images if necessary, so that they will not overflow the page
% margins by default, and it is still possible to overwrite the defaults
% using explicit options in \includegraphics[width, height, ...]{}
\setkeys{Gin}{width=\maxwidth,height=\maxheight,keepaspectratio}
% Set default figure placement to htbp
\makeatletter
\def\fps@figure{htbp}
\makeatother
\setlength{\emergencystretch}{3em} % prevent overfull lines
\providecommand{\tightlist}{%
  \setlength{\itemsep}{0pt}\setlength{\parskip}{0pt}}
\setcounter{secnumdepth}{-\maxdimen} % remove section numbering
\usepackage{amssymb}
\ifLuaTeX
  \usepackage{selnolig}  % disable illegal ligatures
\fi

\begin{document}
\maketitle

\hypertarget{homework-problem-1-functions}{%
\section{Homework Problem 1:
Functions}\label{homework-problem-1-functions}}

In this exercise do not use the built-in functions cov, cor, cov2cor or
any additional R packages.\\

\begin{Shaded}
\begin{Highlighting}[]
\FunctionTok{setwd}\NormalTok{(}\FunctionTok{getwd}\NormalTok{())}
\end{Highlighting}
\end{Shaded}

\hypertarget{a}{%
\subsection{a)}\label{a}}

Create an R function that takes a data matrix
\(\textbf{X} \in R^{n\times p}, n > p\), as an argument and returns the
unbiased estimator of the covariance matrix.

Model reference

\begin{Shaded}
\begin{Highlighting}[]
\NormalTok{A }\OtherTok{\textless{}{-}} \FunctionTok{matrix}\NormalTok{(}\FunctionTok{rnorm}\NormalTok{(}\DecValTok{30}\NormalTok{), }\AttributeTok{ncol =} \DecValTok{3}\NormalTok{)}

\FunctionTok{print}\NormalTok{(}\FunctionTok{cov}\NormalTok{(A))}
\end{Highlighting}
\end{Shaded}

\begin{verbatim}
##           [,1]      [,2]      [,3]
## [1,] 0.7120699 0.2395866 0.4013327
## [2,] 0.2395866 0.8111211 0.2044847
## [3,] 0.4013327 0.2044847 1.0328734
\end{verbatim}

The function for unbiased covariance without using the R library

\begin{Shaded}
\begin{Highlighting}[]
\NormalTok{unbiased\_cov }\OtherTok{\textless{}{-}} \ControlFlowTok{function}\NormalTok{(matrix)\{}
\NormalTok{  average }\OtherTok{\textless{}{-}} \FunctionTok{apply}\NormalTok{(matrix, }\DecValTok{2}\NormalTok{, mean)}
\NormalTok{  centered }\OtherTok{\textless{}{-}}  \FunctionTok{sweep}\NormalTok{(matrix, }\DecValTok{2}\NormalTok{, average, }\StringTok{"{-}"}\NormalTok{)}
\NormalTok{  unbiased\_covariance }\OtherTok{\textless{}{-}} \FunctionTok{t}\NormalTok{(centered) }\SpecialCharTok{\%*\%}\NormalTok{ (centered) }\SpecialCharTok{/}\NormalTok{ (}\FunctionTok{nrow}\NormalTok{(matrix) }\SpecialCharTok{{-}} \DecValTok{1}\NormalTok{)}
  \FunctionTok{return}\NormalTok{(unbiased\_covariance)}
\NormalTok{\}}

\FunctionTok{print}\NormalTok{(}\FunctionTok{unbiased\_cov}\NormalTok{(A))}
\end{Highlighting}
\end{Shaded}

\begin{verbatim}
##           [,1]      [,2]      [,3]
## [1,] 0.7120699 0.2395866 0.4013327
## [2,] 0.2395866 0.8111211 0.2044847
## [3,] 0.4013327 0.2044847 1.0328734
\end{verbatim}

\hypertarget{b}{%
\subsection{b)}\label{b}}

Create an R function that takes a full-rank covariance matrix
\(\textbf{A} \in R^{p\times p}\) as an argument and returns the square
root of the inverse matrix such that
\(\textbf{A} ^ {\frac{-1}{2}} \textbf{A} ^{\frac{-1}{2}} = \textbf{A} ^ {}\)

A full-rank covariance matrix is a square matrix that has linearly
independent columns and rows, which means it has the maximum possible
rank (number of non-zero eigenvalues) which is equal to the number of
rows (or columns) of the matrix. In other words, it is a matrix that has
no zero eigenvalues, which indicates that all its columns are linearly
independent, thus all its rows are also linearly independent.
Additionally, it is a covariance matrix, which means it is symmetric.

Model reference

\begin{Shaded}
\begin{Highlighting}[]
\CommentTok{\# First install and load expm package}
\CommentTok{\# install.packages("expm")}
\FunctionTok{library}\NormalTok{(expm)}
\end{Highlighting}
\end{Shaded}

\begin{verbatim}
## Loading required package: Matrix
\end{verbatim}

\begin{verbatim}
## 
## Attaching package: 'expm'
\end{verbatim}

\begin{verbatim}
## The following object is masked from 'package:Matrix':
## 
##     expm
\end{verbatim}

\begin{Shaded}
\begin{Highlighting}[]
\CommentTok{\# full{-}rank covariance matrix}
\NormalTok{A }\OtherTok{\textless{}{-}} \FunctionTok{matrix}\NormalTok{(}\FunctionTok{c}\NormalTok{(}\FloatTok{0.99}\NormalTok{,}\SpecialCharTok{{-}}\FloatTok{0.17}\NormalTok{,}\FloatTok{0.30}\NormalTok{,}\SpecialCharTok{{-}}\FloatTok{0.17}\NormalTok{,}\FloatTok{0.61}\NormalTok{,}\FloatTok{0.33}\NormalTok{,}\FloatTok{0.30}\NormalTok{,}\FloatTok{0.33}\NormalTok{,}\FloatTok{1.33}\NormalTok{), }\AttributeTok{nrow =} \DecValTok{3}\NormalTok{, }\AttributeTok{ncol =} \DecValTok{3}\NormalTok{)}

\CommentTok{\# Find the inverse of the matrix}
\NormalTok{inverse\_A }\OtherTok{\textless{}{-}} \FunctionTok{solve}\NormalTok{(A)}

\CommentTok{\# Find the square root of the inverse of the matrix}
\NormalTok{sqrt\_inverse\_A }\OtherTok{\textless{}{-}} \FunctionTok{sqrtm}\NormalTok{(inverse\_A)}

\NormalTok{sqrt\_inverse\_A}
\end{Highlighting}
\end{Shaded}

\begin{verbatim}
##            [,1]       [,2]       [,3]
## [1,]  1.0768396  0.2092707 -0.1800484
## [2,]  0.2092707  1.4301472 -0.2629218
## [3,] -0.1800484 -0.2629218  0.9539935
\end{verbatim}

The square root of a matrix can be found by using the eigenvalue
decomposition of the matrix. The eigenvalue decomposition is a way of
decomposing a matrix into a canonical form, where the matrix is
represented as a product of a matrix of eigenvectors and a diagonal
matrix of eigenvalues.\\

The function for the square root of the inverse of a matrix without
using the R library\\

\begin{Shaded}
\begin{Highlighting}[]
\CommentTok{\# Find the square root of the inverse of the full{-}rank covariance matrix}
\NormalTok{sqrt\_inverse }\OtherTok{\textless{}{-}} \ControlFlowTok{function}\NormalTok{(matrix)\{}
  
\NormalTok{  inverse\_matrix }\OtherTok{\textless{}{-}} \FunctionTok{solve}\NormalTok{(matrix)}
  
  \CommentTok{\# Find the eigenvalues and eigenvectors of the matrix}
\NormalTok{  result }\OtherTok{\textless{}{-}} \FunctionTok{eigen}\NormalTok{(inverse\_matrix)}
\NormalTok{  eigenvalues }\OtherTok{\textless{}{-}}\NormalTok{ result}\SpecialCharTok{$}\NormalTok{values}
\NormalTok{  eigenvectors }\OtherTok{\textless{}{-}}\NormalTok{ result}\SpecialCharTok{$}\NormalTok{vectors}
  
  \CommentTok{\# Obtain the square roots ò the eigenvalues}
\NormalTok{  sqrt\_eigenvalues }\OtherTok{\textless{}{-}} \FunctionTok{sqrt}\NormalTok{(eigenvalues)}
  
  \CommentTok{\# Form a diagonal matrix with the square root of eigenvalues}
\NormalTok{  sqrt\_eigenvalues\_matrix }\OtherTok{\textless{}{-}} \FunctionTok{diag}\NormalTok{(sqrt\_eigenvalues)}
  
  \CommentTok{\# Multiply the matrix of eigenvectors with the diagonal matrix of }
  \CommentTok{\# the square roots of the eigenvalues}
\NormalTok{  sqrt\_inverse\_matrix }\OtherTok{\textless{}{-}}\NormalTok{ eigenvectors }\SpecialCharTok{\%*\%}\NormalTok{ sqrt\_eigenvalues\_matrix }\SpecialCharTok{\%*\%} \FunctionTok{t}\NormalTok{(eigenvectors)}
  
  \FunctionTok{return}\NormalTok{(sqrt\_inverse\_matrix)}
\NormalTok{\}}

\FunctionTok{print}\NormalTok{(}\FunctionTok{sqrt\_inverse}\NormalTok{(A))}
\end{Highlighting}
\end{Shaded}

\begin{verbatim}
##            [,1]       [,2]       [,3]
## [1,]  1.0768396  0.2092707 -0.1800484
## [2,]  0.2092707  1.4301472 -0.2629218
## [3,] -0.1800484 -0.2629218  0.9539935
\end{verbatim}

This method will only work if the matrix is symmetric and its
eigenvalues are real numbers, because the square root of real matrix is
only defined for normal matrices. Thankfully, symmetry is guaranteed as
it is a covariance matrix, and its eigenvalues are real numbers because
it is a full-rank matrix.\\

\hypertarget{c}{%
\subsection{c)}\label{c}}

Create an R function that takes a full-rank covariance matrix
\(\textbf{A}\) as an argument and returns the corresponding correlation
matrix\\

Model reference\\

\begin{Shaded}
\begin{Highlighting}[]
\NormalTok{correlation\_matrix }\OtherTok{\textless{}{-}} \FunctionTok{cov2cor}\NormalTok{(A)}

\FunctionTok{print}\NormalTok{(correlation\_matrix)}
\end{Highlighting}
\end{Shaded}

\begin{verbatim}
##            [,1]       [,2]      [,3]
## [1,]  1.0000000 -0.2187592 0.2614435
## [2,] -0.2187592  1.0000000 0.3663728
## [3,]  0.2614435  0.3663728 1.0000000
\end{verbatim}

The function for correlation conversion from full-rank covariance matrix
without using the R library

\begin{Shaded}
\begin{Highlighting}[]
\CommentTok{\# Find the correlation matrix from the full{-}rank covariance matrix}
\NormalTok{correlation }\OtherTok{\textless{}{-}} \ControlFlowTok{function}\NormalTok{(cov\_matrix)\{}
  
  \CommentTok{\# Compute the standard deviations of the variables}
\NormalTok{  sqrt\_diag }\OtherTok{\textless{}{-}} \FunctionTok{diag}\NormalTok{(}\DecValTok{1}\SpecialCharTok{/}\FunctionTok{sqrt}\NormalTok{(}\FunctionTok{diag}\NormalTok{(cov\_matrix)))}
  
  \CommentTok{\# Divide the covariance matrix by the outer product of the standard deviations}
\NormalTok{  cor\_matrix }\OtherTok{\textless{}{-}}\NormalTok{ sqrt\_diag }\SpecialCharTok{\%*\%}\NormalTok{ cov\_matrix }\SpecialCharTok{\%*\%}\NormalTok{ sqrt\_diag}

  \FunctionTok{return}\NormalTok{(cor\_matrix)}
\NormalTok{\}}

\FunctionTok{print}\NormalTok{(}\FunctionTok{correlation}\NormalTok{(A))}
\end{Highlighting}
\end{Shaded}

\begin{verbatim}
##            [,1]       [,2]      [,3]
## [1,]  1.0000000 -0.2187592 0.2614435
## [2,] -0.2187592  1.0000000 0.3663728
## [3,]  0.2614435  0.3663728 1.0000000
\end{verbatim}

Read explanation from here:
\url{https://math.stackexchange.com/questions/186959/correlation-matrix-from-covariance-matrix}

\end{document}
